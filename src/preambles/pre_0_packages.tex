% An alternative to babel for XeLaTeX and LuaLaTeX. [LOAD BEFORE BIBLATEX]
\usepackage{polyglossia}
\setdefaultlanguage[babelshorthands=true]{german}
\ifnumequal{\languageToggle}{1}{\setdefaultlanguage{english}}{}


% Extensive support for hypertext in LaTeX
\usepackage{hyperref}
\hypersetup{
	colorlinks,
	linkcolor={red!40!black},
	urlcolor={blue!90!black},
	citecolor=,
}

% Run a document through LaTeX for syntax checking
% \usepackage{syntonly}

% Draw a page-layout diagram
% \usepackage{showframe}

% Advanced font selection in XeLaTeX and LuaLaTeX
\usepackage{fontspec}

% Reimplementation of and extensions to LaTeX verbatim
\usepackage{verbatim}

% Conditional commands in LaTeX documents
\usepackage{ifthen}

% String manipulation for (La)TeX
\usepackage{xstring}

% Define commands that appear not to eat spaces
\usepackage{xspace}

% Context sensitive quotation facilities
\usepackage{csquotes}

% Typeset source code listings using LaTeX
\usepackage{listings}

% float wrapper for algorithms
\usepackage{algorithm}

% layout for algorithmicx
\usepackage[noend]{algpseudocode}

% AMS mathematical facilities for LaTeX
\usepackage{amsmath}

% TeX fonts from the American Mathematical Society
\usepackage{amssymb}

% AMS-LaTeX commutative diagrams
\usepackage{amscd}

% Typesetting theorems (AMS style)
\usepackage{amsthm}

% Dirac bra-ket and set notations
\usepackage{braket}

% Place lines through maths formulae
\usepackage{cancel}

% Access bold symbols in maths mode
\usepackage{bm}

% Macros for manipulating polynomials
\usepackage{polynom}

% Mathematical tools to use with amsmath
\usepackage{mathtools}

% Enhanced support for graphics
\usepackage{graphicx}

% wrapfig – Produces figures which text can flow around
\usepackage{wrapfig}

% Support for sub-captions
\usepackage{subcaption}

% Intermix single and multiple columns
\usepackage{multicol}

% This package provides macros to insert playing cards, single, or hand, or random-hand, Poker or French Tarot or Uno, from png files.
\usepackage{JeuxCartes}

% Selective filtering of error messages and warnings
\usepackage{silence}
\WarningFilter{latex}{`!h' float specifier changed to `!ht'}
\WarningFilter{latex}{`h' float specifier changed to `ht'}

% Create PostScript and PDF graphics in TeX
\usepackage{tikz}

% Draw people-shaped nodes in TikZ
\usepackage{tikzpeople}

% Martin Vogel's Symbols (marvosym) font
\usepackage{marvosym}

% Some symbols created using TikZ
\usepackage[marvosym]{tikzsymbols}

% Coordinate transformation styles for 3d plotting in TikZ
\usepackage{tikz-3dplot}

% The package provides a PGF/TikZ-based mechanism for drawing linguistic (and other kinds of) trees.
\usepackage{forest}

% Create normal/logarithmic plots in two and three dimensions
\usepackage{pgfplots}
\pgfplotsset{compat=newest}

% Framed environments that can split at page boundaries
\usepackage[framemethod=TikZ]{mdframed}

% Customising captions in floating environments
\usepackage{caption}

% Improved interface for floating objects
\usepackage{float}

% A range of footnote options
\usepackage[hang,flushmargin]{footmisc}

% A generic document command parser
\usepackage{xparse}

% Create PostScript and PDF graphics in TeX
\usepackage{pgf}

% defines \foreach
\usepackage{pgffor}

% This package introduces aliases for counters, that share the same counter register and ‘clear’ list. 
\usepackage{aliascnt}

% Verbatim with URL-sensitive line breaks
\usepackage{url}[hyphens]

% Extra control of appendices
\usepackage[toc,page]{appendix}

% Change the resetting of counters
\usepackage{chngcntr}

% Highly customised stacking of objects, insets, baseline changes, etc
\usepackage{stackengine}

% Horizontally columned lists
\usepackage{tasks}

% Typeset exercises, problems, etc. and their answers
\usepackage{exercise}

% Draw visual representations of matrices in LaTeX
\usepackage{drawmatrix}

% Use Twitter’s open source emoji\left( s through LaTeX commands
\usepackage{twemojis}

% Emoji support in (Lua)LaTeX
\usepackage{emoji}

% Publication quality tables in LaTeX
\usepackage{booktabs}

% A collection of symbols
\usepackage{ifsym}

% Print a coloured contour around text
\usepackage[outline]{contour}

% Driver-independent color extensions for LaTeX and pdfLaTeX
\usepackage{xcolor}

% Graphics package-alike macros for “general” boxes
\usepackage{adjustbox}

% A new interface for environments in LaTeX
\usepackage{environ}

% Easy access to the Lorem Ipsum and other dummy texts
\usepackage{lipsum}

% A package for typesetting epigraphs
\usepackage{epigraph}

% A new bookmark (outline) organization for hyperref
\usepackage{bookmark}

% load package if document contains at least one citation
\iftoggle{citations}{
	% Sophisticated Bibliographies in LaTeX
	\usepackage[maxbibnames=99,sorting=none,style=numeric,backend=biber]{biblatex}
	\addbibresource{\thebibpath} % [PATH]
}{}

% load packge if documentclass is book or article
\ifboolexpr{togl {book} or togl {article}}{
	% Extensive control of page headers and footers in LaTeX2ε
	\usepackage{fancyhdr}
	% Alternative headings for toc/lof/lot
	\usepackage{titletoc}
}{}

% load package if documentclass is not a beamer presentation
\nottoggle{beamer}{
	% Flexible and complete interface to document dimensions
	\usepackage[a4paper,bindingoffset=0mm,left=22mm,right=22mm,top=25mm,bottom=30mm,footskip=14mm]{geometry}
	% This package provides user control over the layout of the three basic list environments: enumerate, itemize and description.
	\usepackage[shortlabels, inline]{enumitem}
	\setlist{nolistsep} % smaller spacing in lists
	% Control table of contents, figures, etc
	\usepackage{tocloft}
	% A package for producing multiple indexes
	\usepackage{imakeidx}
	\makeindex[columns=3, title=Index, intoc]
	% Intelligent cross-referencing. [LOAD AFTER \hypersetup AND amsmath]
	\usepackage[nameinlink,noabbrev,capitalise]{cleveref}
	% Simply changing \parskip and \parindent leaves a layout that is untidy; this package (though it is no substitute for a properly-designed class) helps alleviate this untidiness.
	\usepackage{parskip}
	% Access metadata from the git distributed version control system
}{}

% Completely customisable TOCs
\usepackage{etoc} % [LOAD AFTER tocloft]

% This package supports common layouts for tabular column heads in whole documents, based on one-column tabular environment. In addition, it can create multi-lined tabular cells.
\usepackage{makecell}

% The glossaries package supports acronyms and multiple glossaries, and has provision for operation in several languages
% \usepackage[acronym,nopostdot,nonumberlist,nogroupskip,nomain,style=super]{glossaries}
\usepackage{glossaries}

% This package provides an environment for coloured and framed text boxes with a heading line. 
\usepackage[most]{tcolorbox}
\tcbuselibrary{breakable}

% The package adds PDF support to the landscape environment of package lscape, by setting the PDF /Rotate page attribute. Pages with this attribute will be displayed in landscape orientation by conforming PDF viewers
\usepackage{pdflscape}

% This package is designed to format menu sequences, paths and keyboard shortcuts automatically.
\usepackage{menukeys}

% set the standard "gobble" option to the indent of the first line of the code.
\usepackage{lstautogobble}

% The package allows rows and columns to be coloured, and even individual cells.
\usepackage{colortbl}

% The package generates QR (Quick Response) codes in LATEX, without the need for PSTricks or any other graphical package.
\usepackage{qrcode}

% Create tabular cells spanning multiple rows
\usepackage{multirow}

% The package offers access to the large number of web-related icons provided by the included font.
\usepackage{fontawesome5}

% The package provides a command \forloop for doing iteration in LATEX macro programming.
\usepackage{forloop}

% The package’s principal command, \diagbox, takes two arguments (texts for the slash-separated parts of the box), and an optional argument with which the direction the slash will go, the box dimensions, etc., may be controlled.
\usepackage{diagbox}

% The package provides an \ul (underline) command which will break over line ends; this technique may be used to replace \em (both in that form and as the \emph command), so as to make output look as if it comes from a typewriter. The package also offers double and wavy underlining, and striking out (line through words) and crossing out (/// over words).
\usepackage[normalem]{ulem}

% pdfTEX and LuaTEX support several color stacks. This package shows how a separate color stack can be used for transparency, a property besides color that works across page breaks. If the PDF management is used it can also be used with other engines, but without support for page breaks.
\usepackage{transparent}

% The varwidth environment is superficially similar to minipage, but the specified width is just a maximum value — the box may get a narrower “natural” width.
\usepackage{varwidth}

% Implements a command that causes the commands specified in its argument to be expanded after the current page is output. 
\usepackage{afterpage}
