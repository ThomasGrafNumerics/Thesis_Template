%% mathematics %%
% left ( and right ) round brackets
\newcommand{\lr}[1]{\left(#1\right)}
% left { and right } curly brackets
\newcommand{\lrc}[1]{\left\{#1\right\}}
% left [ and right ] square brackets
\newcommand{\lrs}[1]{\left[#1\right]}
% N (set of natural numbers)
\newcommand{\N}{\mathbb{N}}
% set of units in N = N\{0}
\newcommand{\Nunit}{\N^{\times}}
% Z (set of integers)
\newcommand{\Z}{\mathbb{Z}}
% set of units in Z = Z\{0}
\newcommand{\Zunit}{\Z^{\times}}
% Q (set of rationals)
\newcommand{\Q}{\mathbb{Q}}
% R (set of reals)
\newcommand{\R}{\mathbb{R}}
% set of units in R = R\{0}
\newcommand{\Runit}{\R^{\times}}
% R >= 0 (non negative real numbers)
\newcommand{\Rnn}{\R_{\geq 0}}
% R > 0 (positive real numbers)
\newcommand{\Rp}{\R_{>0}}
% C (set of complex numbers)
\newcommand{\C}{\mathbb{C}}
% set of units in C = C\{0}
\newcommand{\Cunit}{\C^{\times}}
% arg
\newcommand{\Carg}[1]{\text{arg}\lr{#1}}
% cis
\newcommand{\cis}[1]{\text{cis}\lr{#1}}
% K
\newcommand{\K}{\mathbb{K}}
% set of units in K = K\{0}
\newcommand{\Kunit}{\K^{\times}}
% set of units
\newcommand{\unitset}[1]{#1^{\times}}
% real part of a complex number Re(...)
\newcommand{\real}[1]{\text{Re}\lr{#1}}
% imaginary part of a complex number Im(...)
\newcommand{\imag}[1]{\text{Im}\lr{#1}}
% absolute value
\newcommand{\abs}[1]{\left\lvert#1\right\rvert}
% norm
\newcommand{\norm}[1]{\left\lVert#1\right\rVert}
% closed interval [a,b]
\newcommand{\ic}[2]{\lrs{#1, #2}}
% closed / open interval [a,b)
\newcommand{\ico}[2]{\left[#1, #2\right)}
% open / closed interval (a,b]
\newcommand{\ioc}[2]{\left(#1, #2\right]}
% open interval (a,b)
\newcommand{\io}[2]{\lr{#1, #2}}
% double fraction
\newcommand{\doublefraction}[4]{\cfrac{\frac{#1}{#2}}{\frac{#3}{#4}}}
% symbol :=
\newcommand{\defeql}{\mathrel{\mathop:}=}
% symbol =:
\newcommand{\defeqr}{=\mathrel{\mathop:}}
% ceiling operator
\newcommand{\ceil}[1]{\left\lceil #1 \right\rceil}
% floor operator
\newcommand{\floor}[1]{\left\lfloor #1 \right\rfloor}
% text italics and bold font
\newcommand{\tib}[1]{\textbf{\textit{#1}}}
% set (condition is in math mode): {n\in\N ; n < 10}
\newcommand{\setcm}[2]{\lrc{\;#1 \; ; \; #2 \;}}
% set (condition is in text mode): {n\in\N ; $n$ is even}
\newcommand{\setct}[2]{\lrc{\;#1 \; ; \; \text{#2} \;}}
% set (all in text mode): {$n$ is a natural number ; $n$ is even}
\newcommand{\settt}[2]{\lrc{\; \text{#1} \; ; \; \text{#2} \;}}
% replace this (not necessary, but used in legacy code)
\newcommand{\elem}[1]{\xrightarrow{\substack{#1}}}
% roman numerals
\newcommand{\romanNumeral}[1]{%
	\textup{\uppercase\expandafter{\romannumeral#1}}%
}

%% code / programming %%
% Python inline
\newcommand{\pythoninline}[1]{{\lstinline[language=Python,basicstyle=\ttfamily\normalsize]{#1}}}
% C++ inline
\newcommand{\cppinline}[1]{{\lstinline[language=C++,basicstyle=\ttfamily\normalsize]{#1}}}
