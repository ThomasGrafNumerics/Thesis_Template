\usetikzlibrary{
  3d,
  arrows,
  arrows.spaced,
  arrows.meta,
  bending,
  babel,
  calc,
  fit,
  hobby,
  patterns,
  patterns.meta,
  plotmarks,
  shapes.geometric,
  shapes.misc,
  shapes.symbols,
  shapes.arrows,
  shapes.callouts,
  shapes.multipart,
  shapes.gates.logic.US,
  shapes.gates.logic.IEC,
  circuits.logic.US,
  circuits.logic.IEC,
  circuits.logic.CDH,
  circuits.ee.IEC,
  datavisualization,
  datavisualization.polar,
  datavisualization.formats.functions,
  er,
  automata,
  backgrounds,
  chains,
  topaths,
  trees,
  petri,
  mindmap,
  matrix,
  calendar,
  folding,
  fadings,
  shadings,
  spy,
  through,
  turtle,
  positioning,
  scopes,
  decorations.fractals,
  decorations.shapes,
  decorations.text,
  decorations.pathmorphing,
  decorations.pathreplacing,
  decorations.footprints,
  decorations.markings,
  shadows,
  lindenmayersystems,
  intersections,
  fixedpointarithmetic,
  fpu,
  svg.path,
  external,
  graphs,
  graphs.standard,
  quotes,
  math,
  angles,
  views,
  animations,
  rdf,
  perspective,
  tikzmark,
  optics,
  overlay-beamer-styles
}

\tikzstyle{code} = [
rectangle,
rounded corners,
minimum width=3cm,
minimum height=1cm,
text centered,
draw=black,
fill=red!30
]

\tikzstyle{decision} = [
diamond,
aspect=2,
minimum width=3cm,
minimum height=1cm,
text centered,
draw=black,
fill=green!30
]

\tikzstyle{arrow} = [thick,->,>=stealth]
\tikzset{>=latex}
\tikzset{boximg/.style={remember picture,red,thick,draw,inner sep=0pt,outer sep=0pt}}

% command for annotating multiple lines in a code via curly braces
% arg1: xshift (0em for no shift)
% arg2: coord1 from which to draw brace
% arg3: coord2 to which to draw brace
% arg4: annotation text
% arg5: curly brace aspect ratio (0.5 for "normal" braces)
% arg6: rotation (empty for right, "mirror" for left)
% arg7: text shift (default 5em)
\newcommand*{\drawBrace}[7]{%
	
	% fit rectangle
    \node[fill=none, draw=none, fit={(#2) (#3)}, inner sep=0pt] (rectg) {};%
    
    % curly brace
    \draw [decoration={brace,#6,amplitude=0.3em,aspect=#5},decorate,very thick,red]%
      ([xshift=#1]rectg.north east) --%
      coordinate[right=1em, midway] (#2#3)
      ([xshift=#1]rectg.south east);%
    
    % coordinate for arrow start
   	\coordinate (#2#3-coord-comment-y) at ($(#2.north)!#5!(#3.south)$);
   	% coordinate for arrow end
   	\ifstrequal{#6}{mirror}{
   	\def\adjarrow{-.3em}
   	}{
   	\def\adjarrow{.8em}
   	}
   	
    \coordinate (#2#3-coord-arrow) at ($(#2#3-coord-comment-y)+(#1,0)+(\adjarrow,0)$);
    
    % add text (if provided)
    \ifblank{#4}{}{      
        % comment
        \ifblank{#7}{
        \node[right=5em of #2#3-coord-comment-y](#2#3-comment) {#4};
        }{
        \node[right=#7 of #2#3-coord-comment-y](#2#3-comment) {#4};
        }
        
        
        % arrow pointing from comment to brace
        \draw[->, style={thick, >=stealth, red!50}] (#2#3-comment.west) to (#2#3-coord-arrow);
    }  
   	
}%

% command for annotating a line in a code via an arrow
% arg1: code mark (using tikzmark, "to")
% arg2: comment mark (using tikzmark, "from")
% arg3: annotation text
% arg4: fill and arrow color
\newcommand*{\drawArrow}[4]{%
    % comment
    \node[draw=#4, fill=#4](#2-comment) at(#2.east) {#3};
    
    % arrow pointing from comment to brace
    \draw[->,style=#4] (#2-comment.west) -- (#1);
}%

\renewcommand{\tikzmark}[2][]{%
  \tikz[remember picture,overlay,baseline=-.5ex] \node[#1] (#2) {};%
}


\newcommand{\markangle}[6][0.25]{
  \begin{scope}
    \clip (#2)--(#3)--(#4);
    \node (ANG) [draw, circle, minimum size=#5] at (#3){};
  \end{scope}
  \coordinate (E1) at (intersection 0 of ANG and #3--#2);
  \coordinate (E2) at (intersection 0 of ANG and #3--#4);
  \coordinate (EM) at ($ (E1)!0.5!(E2) $);
  \coordinate (EM) at ($ (EM)!#1!(#3) $);
  \node at (EM) {#6};
}
