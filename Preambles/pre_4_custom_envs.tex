\nottoggle{beamer}{
	% enumerate with roman numbers
	\newlist{renum}{enumerate}{1}
	\setlist[renum,1]{label=(\roman*)}

	% enumerate alphabetically
	\newlist{aenum}{enumerate}{1}
	\setlist[aenum,1]{label=(\alph*)}
} {%
}

\newtheoremstyle{mytheoremstyle}% 〈name〉
{6pt}%  <Space above>
{6}%  <Space below>
{\normalfont}%  <Body font〉
{}% <Indent amount>
{\bfseries}% <Theorem head font〉
{:}% Punctuation after theorem head〉
{\newline}% Space after theorem head 
{}% <Theorem head spec (can be left empty, meaning ‘normal’)〉

\theoremstyle{mytheoremstyle}

\iftoggle{book}{%
	\iftoggle{german}{%
		\newtheorem{myexample}{Beispiel}[chapter]
		\newtheorem{myremark}{Bemerkung}[chapter]
		\newtheorem{mydefinition}{Definition}[chapter]
		\newtheorem{myaxiom}{Axiom}[chapter]
		\newtheorem{mytheorem}{Theorem}[chapter]
		\newtheorem{myproof}{Beweis}[chapter]
		\newtheorem{mysummary}{Zusammenfassung}[chapter]
		\newtheorem{myprocedure}{Vorgehen}[chapter]
	} {%
		\newtheorem{myexample}{example}[chapter]
		\newtheorem{myremark}{remark}[chapter]
		\newtheorem{mydefinition}{definition}[chapter]
		\newtheorem{myaxiom}{axiom}[chapter]
		\newtheorem{mytheorem}{theorem}[chapter]
		\newtheorem{myproof}{proof}[chapter]
		\newtheorem{mysummary}{summary}[chapter]
		\newtheorem{myprocedure}{procedure}[chapter]
	}%
} {%
	\iftoggle{german}{%
		\newtheorem{myexample}{Beispiel}[section]
		\newtheorem{myremark}{Bemerkung}[section]
		\newtheorem{mydefinition}{Definition}[section]
		\newtheorem{myaxiom}{Axiom}[section]
		\newtheorem{mytheorem}{Theorem}[section]
		\newtheorem{myproof}{Beweis}[section]
		\newtheorem{mysummary}{Zusammenfassung}[section]
		\newtheorem{myprocedure}{Vorgehen}[section]
	} {%
		\newtheorem{myexample}{example}[section]
		\newtheorem{myremark}{remark}[section]
		\newtheorem{mydefinition}{definition}[section]
		\newtheorem{myaxiom}{axiom}[section]
		\newtheorem{mytheorem}{theorem}[section]
		\newtheorem{myproof}{proof}[section]
		\newtheorem{mysummary}{summary}[section]
		\newtheorem{myprocedure}{procedure}[section]
	}%
}%

\tcolorboxenvironment{myexample}{
	enhanced jigsaw,colframe=DarkSeaGreen,interior hidden,
	breakable,before skip=10pt,after skip=10pt }
\tcolorboxenvironment{myremark}{
	enhanced jigsaw,colframe=Indigo,interior hidden,
	breakable,before skip=10pt,after skip=10pt }
\tcolorboxenvironment{mydefinition}{
	enhanced jigsaw,colframe=Goldenrod,interior hidden,
	breakable,before skip=10pt,after skip=10pt }
\tcolorboxenvironment{myaxiom}{
	enhanced jigsaw,colframe=Tomato,interior hidden,
	breakable,before skip=10pt,after skip=10pt }
\tcolorboxenvironment{mytheorem}{
	enhanced jigsaw,colframe=Teal,interior hidden,
	breakable,before skip=10pt,after skip=10pt }
\tcolorboxenvironment{myproof}{% `proof' from `amsthm'
	blanker,breakable,left=5mm,
	before skip=10pt,after skip=10pt,
	borderline west={1mm}{0pt}{Magenta}}
\tcolorboxenvironment{mysummary}{
	enhanced jigsaw,colframe=MintCream,interior hidden,
	breakable,before skip=10pt,after skip=10pt }
\tcolorboxenvironment{myprocedure}{
	enhanced jigsaw,colframe=Lavender,interior hidden,
	breakable,before skip=10pt,after skip=10pt }

% tcolorbox baseline template
\newtcolorbox{tcolorboxTemplate}[1][]{
	enhanced jigsaw,
	lines before break=9,
	breakable,
	parbox=false,
	attach boxed title to top text left={yshift=-3mm,yshifttext=-3mm},
	#1}

% mybox
\newenvironment{mybox}
{%
	\begin{tcolorboxTemplate}[
			colback=Khaki!10, % Background color
			coltitle=white,
			colframe=Khaki!50!black,
			colbacktitle=Khaki!50!black,
			title=\faExclamationTriangle{} Box]
		}%%
		{%
	\end{tcolorboxTemplate}
}%

% mycaution
\newenvironment{mycaution}
{%
	\begin{tcolorboxTemplate}[
			colback=Red!10, % Background color
			coltitle=white,
			colframe=Red!50!black,
			colbacktitle=Red!50!black,
			title=\faExclamationTriangle{} Achtung]
		}%%
		{%
	\end{tcolorboxTemplate}
}%

% counter shared between 'myexercise', 'mychallenge' and 'mysubmission'
\iftoggle{book} {%
	\newcounter{SharedCounter}[chapter]
} {%
	\newcounter{SharedCounter}[section]
}

\setcounter{SharedCounter}{0}

\iftoggle{book} {%
	\renewcommand{\theSharedCounter}{\arabic{chapter}.\arabic{SharedCounter}}
} {%
	\renewcommand{\theSharedCounter}{\arabic{section}.\arabic{SharedCounter}}
}

% myexercise
\newaliascnt{ExerciseCounter}{SharedCounter}
\aliascntresetthe{ExerciseCounter}
\newenvironment{myexercise}[1][]
{%
	\refstepcounter{ExerciseCounter}
	\begin{tcolorboxTemplate}[
			title=\faEdit{} Aufgabe~\theSharedCounter~#1,
			colback=RoyalBlue!10,
			colframe=RoyalBlue,
			colbacktitle=RoyalBlue]
		}%%
		{%
	\end{tcolorboxTemplate}
}%

% mysubmission
\newaliascnt{SubmissionCounter}{SharedCounter}
\aliascntresetthe{SubmissionCounter}
\newenvironment{mysubmission}[1][]
{%
	\refstepcounter{SubmissionCounter}
	\begin{tcolorboxTemplate}[
			title=\faArrowUp{} Aufgabe (Abgabe)~\theSharedCounter~#1,
			colback=Purple!10,
			colframe=Purple,
			colbacktitle=Purple]
		}%%
		{%
	\end{tcolorboxTemplate}
}%

% mychallenge
\newaliascnt{ChallengeCounter}{SharedCounter}
\aliascntresetthe{ChallengeCounter}
\newenvironment{mychallenge}[1][]
{%
	\refstepcounter{ChallengeCounter}
	\begin{tcolorboxTemplate}[
			title=\faTrophy{} Aufgabe (Challenge)~\theSharedCounter~#1,
			colback=Gold!10,
			colframe=Gold!50!black,
			colbacktitle=Gold!50!black]
		}%%
		{%
	\end{tcolorboxTemplate}
}%

% myanswer
% \begin{@empty}
% \end{@empty}

\newcounter{DummyCounter}
\newcommand{\myanswerlabel}{-}
\newenvironment{myanswer}[1][]
{
\refstepcounter{DummyCounter}

\ifblank{#1} {%
	\renewcommand{\myanswerlabel}{Aufgabe~\theSharedCounter}
	\renewcommand{\AnswerHeader}{}
	\newcommand{\myanswercapsule}{tcolorboxTemplate}
} {%
	\renewcommand{\myanswerlabel}{\cref{#1}}
	\renewcommand{\AnswerHeader}{
		\begin{center}
			\textcolor{ForestGreen}{Lösungsvorschlag zu}~\myanswerlabel
		\end{center}
	}
	\newcommand{\myanswercapsule}{@empty}
}%
\begin{Answer}[ref=\theDummyCounter]
\ifblank{#1} {%
	\begin{\myanswercapsule}[
		opacityback=.5,
		coltitle=white,
		colback=ForestGreen!10,
		colframe=ForestGreen,
		colbacktitle=ForestGreen,
		opacityback=1,
		title=\faCheck{}~Lösungsvorschlag zu~\myanswerlabel]
} {
	\begin{\myanswercapsule}
		}%
		}%
		{%
	\end{\myanswercapsule}
	\end{Answer}
}%

% names for clever references
\nottoggle{beamer}{%
	\iftoggle{german}{%
		\crefname{ExerciseCounter}{Aufgabe}{Aufgaben}
		\Crefname{ExerciseCounter}{Aufgabe}{Aufgaben}
		\crefname{SubmissionCounter}{Abgabe}{Abgaben}
		\Crefname{SubmissionCounter}{Abgabe}{Abgaben}
		\crefname{ChallengeCounter}{Challenge}{Challenges}
		\Crefname{ChallengeCounter}{Challenge}{Challenges}
		\crefname{myexample}{Beispiel}{Beispiele}
		\crefname{myremark}{Bemerkung}{Bemerkungen}
		\crefname{mydefinition}{Definition}{Definitionen}
		\crefname{myaxiom}{Axiom}{Axiome}
		\crefname{mytheorem}{Theorem}{Theoreme}
		\crefname{myproof}{Beweis}{Beweise}
		\crefname{mysummary}{Zusammenfassung}{Zusammenfassungen}
		\crefname{myprocedure}{Vorgehen}{Vorgehen}
	} {%
		\crefname{ExerciseCounter}{exercise}{exercise}
		\Crefname{ExerciseCounter}{Exercise}{Exercises}
		\crefname{SubmissionCounter}{submission}{submissions}
		\Crefname{SubmissionCounter}{submission}{Submissions}
		\crefname{ChallengeCounter}{challenge}{challenges}
		\Crefname{ChallengeCounter}{Challenge}{Challenges}
		\crefname{myexample}{example}{examples}
		\crefname{myremark}{remark}{remarks}
		\crefname{mydefinition}{definition}{definitions}
		\crefname{myaxiom}{axiom}{axioms}
		\crefname{mytheorem}{theorem}{theorems}
		\crefname{myproof}{proof}{proofs}
		\crefname{mysummary}{summary}{summaries}
		\crefname{myprocedure}{procedure}{procedures}
	}%
}{}
