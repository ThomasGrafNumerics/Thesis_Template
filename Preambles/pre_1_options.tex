%% settings for listings-package %%
% basline listings style
\lstdefinestyle{BaselineStyle}
{ %
	commentstyle=\color{Blue},
	keywordstyle=\color{Green},
	numberstyle=\tiny\color{Gray},
	stringstyle=\color{Fuchsia},
	basicstyle=\ttfamily,
	columns=flexible,
	breakatwhitespace=false,
	breaklines=true,
	captionpos=b,
	keepspaces=true,
	numbers=left,
	numbersep=5pt,
	showspaces=false,
	showstringspaces=false,
	showtabs=false,
	tabsize=4,
	xleftmargin=1.5em,
	frame=single,
	framexleftmargin=1.5em,
	autogobble=true
} %
% set to 'BaselineStyle'
\lstset{style=BaselineStyle}

% our Python style
\lstdefinestyle{Python}
{ %
	language=Python,
	backgroundcolor=\color{Goldenrod!5},
	frame=none,
	framexleftmargin=0em,
	xleftmargin=0em,
	belowskip=0pt
} %

\lstdefinestyle{sql}
{ %
	%	backgroundcolor=\color{BlanchedAlmond!20},
	language=sql,
	frame=lines,
	numbers=none,
	deletekeywords={ROLE},
	morekeywords={LENGTH,IS,RENAME,TO,REFERENCES,ENUM,TINYINT,DATETIME,AUTO_INCREMENT,UNSIGNED,REAL,TEXT,BOOL,BOOLEAN, IF},
	linewidth=\linewidth,
	framexleftmargin=0em,
	xleftmargin=0em,
	belowskip=0pt
} %

% to show a LaTeX source code and its compiled resulting document in a tcolorbox
\tcbuselibrary{listings}

%% specific settings for different document classes %%
\iftoggle{exam}{
	%% EXAM ENVIRONMENT %%
	\iftoggle{german}{
		\pointpoints{Punkt}{Punkte}
		\bonuspointpoints{Bonuspunkt}{Bonuspunkte}
		\renewcommand{\solutiontitle}{\noindent\textbf{Lösungsvorschlag:}\enspace}
		\hqword{Aufgabe:}
		\hpgword{Seite:}
		\hpword{Punkte:}
		\hsword{erhalten:}
		\htword{Total}
		\vqword{Frage}
		\bhqword{Aufgabe:}
		\bhpgword{Seite:}
		\bhpword{Bonuspunkte:}
		\bhsword{erhalten:}
		\bhtword{Total}
		\chqword{Aufgabe:}
		\chpgword{Seite:}
		\chpword{Punkte:}
		\chbpword{Bonuspunkte:}
		\chsword{erhalten:}
		\chtword{Total}
		\totalformat{Aufgabe \thequestion\ total: \totalpoints\ Punkte}
	}{}
	%\checkedchar{\CheckedBox}
	\runningheadrule
	% \firstpageheader{\mycourse}{\myclass}{\mydate}
	\runningheader{\mycourse}{\myclass}{\mydate}
	\firstpagefooter{}{\thepage\,/\,\numpages}{}
	\runningfooter{}{\thepage\,/\,\numpages}{}
	\CorrectChoiceEmphasis{\normalfont}
	% \renewcommand{\questionlabel}{\bfseries Aufgabe~\thequestion.}
}{}

\iftoggle{book}{
	\setcounter{secnumdepth}{5}
	\setcounter{tocdepth}{5}
	% start chapter numbering at different number (set to -1 to start at 0)
	\setcounter{chapter}{0}
	% slightly changing chapter marking in header
	\renewcommand{\chaptermark}[1]{\markboth{\textnormal{\thechapter}\ \textnormal{#1}}{}}
	% slightly changing section marking in header
	\renewcommand{\sectionmark}[1]{\markright{\textnormal{\thesection}\ \textnormal{#1}}{}}
}{}

% BEAMER %
\iftoggle{beamer}{
	\beamertemplatenavigationsymbolsempty
}{}

\iftoggle{german}{
	%% replace English terms with German counterparts %%
	\renewcommand{\lstlistingname}{Programm}
	\floatname{algorithm}{Algorithmus}
	\renewcommand{\algorithmicrequire}{\textbf{Eingabe:}}
	\renewcommand{\algorithmicensure}{\textbf{Resultat:}}

	%% clever refs %%
	\nottoggle{beamer}{
		\crefname{listing}{Programm}{Programme}
		\Crefname{listing}{Programm}{Programme}
	}{}
}{}
